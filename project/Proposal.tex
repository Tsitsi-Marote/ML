%%%%%%%%%%%%%%%%%%%%%%%%%%%%%%%%%%%%%%%%%
% University Assignment Title Page 
% LaTeX Template
% Version 1.0 (27/12/12)
%
% This template has been downloaded from:
% http://www.LaTeXTemplates.com
%
% Original author:
% WikiBooks (http://en.wikibooks.org/wiki/LaTeX/Title_Creation)
%
% License:
% CC BY-NC-SA 3.0 (http://creativecommons.org/licenses/by-nc-sa/3.0/)
% 
% Instructions for using this template:
% This title page is capable of being compiled as is. This is not useful for 
% including it in another document. To do this, you have two options: 
%
% 1) Copy/paste everything between \begin{document} and \end{document} 
% starting at \begin{titlepage} and paste this into another LaTeX file where you 
% want your title page.
% OR
% 2) Remove everything outside the \begin{titlepage} and \end{titlepage} and 
% move this file to the same directory as the LaTeX file you wish to add it to. 
% Then add \input{./title_page_1.tex} to your LaTeX file where you want your
% title page.
%
%%%%%%%%%%%%%%%%%%%%%%%%%%%%%%%%%%%%%%%%%
%\title{Title page with logo}
%----------------------------------------------------------------------------------------
%	PACKAGES AND OTHER DOCUMENT CONFIGURATIONS
%----------------------------------------------------------------------------------------

\documentclass[12pt]{article}
\usepackage[english]{babel}
\usepackage[utf8x]{inputenc}
\usepackage{amsmath}
\usepackage{graphicx}
\usepackage[colorinlistoftodos]{todonotes}

\begin{document}

\begin{titlepage}

\newcommand{\HRule}{\rule{\linewidth}{0.5mm}} % Defines a new command for the horizontal lines, change thickness here

\center % Center everything on the page
 
%----------------------------------------------------------------------------------------
%	HEADING SECTIONS
%----------------------------------------------------------------------------------------

\textsc{\LARGE University of the Witwatersrand}\\[1cm] % Name of your university/college
%\textsc{\Large Major Heading}\\[0.5cm] % Major heading such as course name
%\textsc{\large Minor Heading}\\[0.5cm] % Minor heading such as course title

%----------------------------------------------------------------------------------------
%	TITLE SECTION
%----------------------------------------------------------------------------------------

\HRule \\[0.4cm]
{ \huge \bfseries Supervised Learning for acne Tracking Application}\\[0.4cm] % Title of your document
\HRule \\[1.5cm]
 
%----------------------------------------------------------------------------------------
%	AUTHOR SECTION
%----------------------------------------------------------------------------------------

\begin{minipage}{0.4\textwidth}
\begin{flushleft} \large
\emph{Author:}\\
Tsitsi \textsc{Marote}\\ 856182 % Your name
\end{flushleft}
\end{minipage}
~
\begin{minipage}{0.4\textwidth}
\begin{flushright} \large
\emph{Author:} \\
Meriam \textsc{Elabor} \\1076589 % Supervisor's Name
\end{flushright}
\end{minipage}\\[2cm]

% If you don't want a supervisor, uncomment the two lines below and remove the section above
%\Large \emph{Author:}\\
%John \textsc{Smith}\\[3cm] % Your name

%----------------------------------------------------------------------------------------
%	DATE SECTION
%----------------------------------------------------------------------------------------

%{\large \today}\\[2cm] % Date, change the \today to a set date if you want to be precise

%----------------------------------------------------------------------------------------
%	LOGO SECTION
%----------------------------------------------------------------------------------------

\includegraphics{logo.jpeg}\\[1cm] % Include a department/university logo - this will require the graphicx package
 
%----------------------------------------------------------------------------------------

\vfill % Fill the rest of the page with whitespace

\end{titlepage}


\section*{Project Description}
The aim of this report is to investigate and compare various supervised learning techniques applied to the problem of acne detection and classification \cite{mobile}. \\ \medskip 

The process of counting and classifying acne lesions is an important part of the treatment process and can be tedious and may yield inaccurate results if not done properly. As a dermatological problem that affects many people\cite{pathogens}, we propose a mobile application that will be able to accurately track the progress of the user. Acne lesions can be classified into several skin types, including comedone, pustule, reddish papule, with or scarring without. \cite{mobile} As such, speed and image resolution will be a significant constraint. \\ \medskip

Much work has been done to solve this problem with various computational techniques\cite{acne1, acne2, acne3, acne4}. As such, we will consider different types of features and multiple approaches to extract features and to count an classify individual lesions. This process will most likely involve skin detection\cite{skin1, skin2} and other computer vision techniques as a part of preprocessing. \\ \medskip


%\section{Application of Reinforcement Learning in Pacman}
%The aim of this research is to implement reinforcement learning to ensure that the pacman makes its own decisions throughout the game. This technique trains the game based on the rewards it gets for moves\cite{Games}. Points will be used to train pacman to know when a move is a good or bad move. Even when multiple moves have been done before getting a reward, pacman still has to be able to tell which moves were good and which ones were bad\cite{Games}. 

\section*{Deliverables}
As a part of the report, the following will be produced amongst others:
\begin{itemize}
\item A suitable skin detection algorithm.
\item A database of labeled images of acne patients referenced from \cite{data}
\item Various methods to recognize and classify acne lesions
\item Comparison of these methods performace
\item A mobile application utilizing the most appropriate method. \cite{*}
\end{itemize}

\bibliographystyle{ieeetr}
\bibliography{ml}


\end{document}